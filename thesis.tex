\documentclass[oneside,senior,etd]{BYUPhys}

\usepackage[utf8]{inputenc}
\usepackage{rotating}

\usepackage[russian]{babel}
\usepackage{amsfonts} % Пакеты для математических символов и теорем
\usepackage{amstext}
\usepackage{amssymb}
\usepackage{amsthm}
\usepackage{graphicx} % Пакеты для вставки графики
\usepackage{subfig}
\usepackage{color}
\usepackage[unicode]{hyperref}
\usepackage[nottoc]{tocbibind} % Для того, чтобы список литературы отображался в оглавлении
\usepackage{verbatim} % Для вставок заранее подготовленного текста в режиме as-is
\usepackage{listings}

\newcommand{\sectionbreak}{\clearpage} % Раздел с новой станицы

\usepackage{tikz}
\usepackage{pgfplots}
\usetikzlibrary{arrows,positioning}
\usepackage{adjustbox}

\usepackage{makecell}
\usepackage{booktabs}
\usepackage{boldline}

\usepackage{xcolor}
\usepackage{soul}
\usepackage{url}
\usepackage{multirow}
\usepackage{amsmath}

\usepackage{pifont}

\newcommand{\todo}[1]{\textcolor{red}{#1}}
\newcommand*{\R}{\mathbb R}

{
\Faculty{Факультет вычислительной математики и кибернетики}
\Chair{Кафедра математической статистики}
% Лаборатория
% \Lab{~}
\Year{2020}
  \Month{Месяц}
  \City{Москва}
  \AuthorText{Автор:}
  \Author{Осипова Анастасия Андреевна}
  \AuthorEng{Anastasiya Osipova}
  \AcadGroup{620}

  \TitleTop{Адаптивные статистические алгоритмы оценивания параметров}
  \TitleMiddle{конечных смешанных нормальных моделей}
  % \TitleBottom{Третья строка названия}
  %\TitleTopEng{Thesis theme, first line}
  % Uncomment if you need more title lines
  % \TitleMiddleEng{Thesis theme, second line}
  % \TitleBottomEng{Thesis theme, third line}

\docname{ВВЕДЕНИЕ В МАГИСТЕРСКУЮ ДИССЕРТАЦИЮ}
  \Advisor{Королев Виктор Юрьевич}
  \AdvisorDegree{д.ф-м.н., профессор\\}
  % Раскомментируйте, чтобы добавить научного консультанта
  \Consultant{Горшенин Андрей Константинович}
  \ConsultantDegree{к.ф-м.н., доцент\\}

% Закомментируйте, если аннотация не нужна
%\Abstract{
%Аннотация.
%}
% Раскомментируйте, если нужна английская аннотация
% \AbstractEng{Abstract in English}

% Раскомментируйте, чтобы написать благодарности
% \Acknowledgments{Благодарности.}

%%%% DON'T change this. It is here because .sty does not support cyrillic cp properly %%%%
\TitlePageText{Титульная страница}
\University{Московский государственный университет имени М.В.Ломоносова}
\GrText{гр.}
\AdvisorText{Научный руководитель}
\ConsultantText{Научный консультант}
\AbstractText{Аннотация}
\AcknowledgmentsText{Благодарности}
\ListingText{Листинг}
\AlgorithmText{Алгоритм}

% Set PDF title and author
\hypersetup{
  pdftitle={\PDFTitle},
  pdfauthor={\PDFAuthor}
}
}

\begin{document}
{
\fixmargins
 \makepreliminarypages
\oneandhalfspace
\pdfbookmark[section]{\contentsname}{toc}
\tableofcontents
}
\section{Введение}
	Метод скользящего разделения смесей (СРС-метод) является развитием идеи EM-алгоритма, применяемого в статистике для поиска оценок максимального правдоподобия в случаях, когда целевая функция правдоподобия имеет сложную структуру. СРС-метод позволяет в динамическом режиме выделять компоненты смеси и применяется в таких областях, как физика турбулентной плазмы \cite{korolev2011book}, обработка данных финансовых рынков \cite{skvortsova2006estimation}, анализ потоков тепла между океаном и атмосферой \cite{gorshenin2020stat}.
	\\
	
	Данная работа состоит из трех логических блоков, каждому из которых посвящена отдельная глава. В первом блоке рассматривается адаптивная модификация СРС-метода, изучается ее точность и условия применимости для различных взаимных вариантов расположения компонент сигнала и шума. Также представлен алгоритм автоматического определения момента разладки (момента, когда меняется структура данных и в записи помимо шума оборудования появляется полезный сигнал).
	\\
	
	Во втором блоке предлагается алгоритм выделения компонент связности после обработки данных СРС-методом для восстановления "истории"\ развития каждой из компонент и демонстрируется его применение к данным по потокам тепла в различных климатических зонах.
	\\
	
	В третьем блоке рассматривается возможность использования метода для улучшения точности прогнозирования временных рядов путем расширения признакового пространства информацией об эволюции компонент. Для применения этого подхода требуется лишь предположение о подчинении данных рассматриваемого временного ряда структуре конечной смеси нормальных законов, что является достаточно широко используемой моделью описания данных, например, в области эволюции финансовых потоков.

	
\section{Постановка задачи}
	Требуется провести широкий анализ практического применения адаптивного метода выделения сигнала на фоне зашумленных данных. Можно выделить три основных части, в каждой из которых решается отдельная подзадача, связанная с данным методом: 
	\begin{itemize}
		\item Анализ эффективности адаптивного метода скользящего разделения смесей (адаптивного СРС-метода)
		\item Выделение компонент связности из оценок, полученных алгоритмом
		\item Расширение пространства признаков при построении нейронных сетей
	\end{itemize}
	
	Каждая из подзадач является достаточно самостоятельной и полноценной задачей, поэтому под каждую из них отведён отдельный раздел, где содержится математическая постановка задачи, описание решения и полученные результаты, а для второй и третьей подзадач -- связь с предыдущими этапами.

\section{Анализ эффективности адаптивного метода}
	На первом этапе работы была продемонстрирована эффективность предложенной в работе~\cite{gorshenin2019adaptive} процедуры определения параметров полезного сигнала при условии наличия шума для различных соотношений между их параметрами на примере $24$ модельных выборок, охватывающих большинство возможных реальных сценариев. Также в данном разделе обсуждаются вопросы прикладного подхода к обнаружению момента появления полезного сигнала в наблюдениях.
	\\
	
	Результаты работы по данному этапу были опубликованы в статье \cite{gorshenin2020efficiency}.
	
	\subsection{Предпосылки}
		Наблюдения (сигналы) в реальных системах зачастую регистрируются с округлениями и дополнительной шумовой составляющей, которая возникает из-за случайных флуктуаций в работе экспериментального оборудования или внешних факторов. Очевидно, что такие модификации получаемой выборки не связаны непосредствено с проводимым экспериментом, однако влияют на его результаты. Данная проблема характерна для широкого спектра исследовательских задач, в том числе в медицинских приложениях~\cite{marquez2020optimal, almgren2020effect}, при анализе сигналов с негауссовским шумом~\cite{asadi2018signal, ilter2019joint, guo2020enhanced}, предобработке изображений~\cite{li2019noise}.
		\\
		
		Особенности работы с округленными данными изучались в статье~\cite{gorshenin2018data}. Для учета влияния случайного шума в статье~\cite{gorshenin2019adaptive} была предложена модель для исходных наблюдений на основе случайной величины (с.в.) $Z$, которая может быть представлена в виде суммы независимых с.в. $X$ (полезный сигнал) и $Y$ (аддитивный шум) с различными смешанными конечными нормальными распределениями. Предполагается, что до начала эксперимента может быть получена выборка реализаций только с.в. $Y$ достаточно большого объема. Данное требование не является ограничительным, поскольку обычно основные сложности регистрации связаны непосредственно с экспериментом (ограниченное время наблюдения за процессом, разрешающая способность оборудования и т.п.), в то время как предварительный запуск детектирующих приборов и получение данных с них являются достаточными простыми процедурами. Оценивание параметров проводится в режиме сдвигающегося окна с помощью метода скользящего разделения смесей~\cite{korolev2011probabilistic}.
	
		
	\subsection{Постановка задачи}
		Обозначим $X$ -- полезный сигнал, $Y$ -- аддитивный шум, $Z=X+Y$ -- наблюдаемые в эксперименте величины:
		\begin{gather}
		\label{XDistr}
		X\sim \sum\limits_{j=1}^{k} p_j \Phi\left(\frac{x-a_j}{\sigma_j}\right),\\
		\text{где } \sum\limits_{j=1}^{k}p_j=1, \, p_j\geqslant0, \, a_j\in\R, \, \sigma_j>0, \, j=\overline{1,k}, \notag\\
		\label{YDistr}
		Y\sim \sum\limits_{j=1}^{\widetilde{k}} \widetilde{p}_j \Phi\left(\frac{x-\widetilde{a}_j}{\widetilde{\sigma}_j}\right),\\
		\text{где } \sum\limits_{j=1}^{\widetilde{k}}\widetilde{p}_j=1, \, \widetilde{p}_j\geqslant0, \, \widetilde{a}_j\in\R, \, \widetilde{\sigma}_j>0, \, j=\overline{1,\widetilde{k}}, \notag\\
		\label{ZDistr}
		Z\sim \sum\limits_{l=1}^{k\cdot\widetilde{k}} \widehat{p}_l \Phi\left(\frac{x-\widehat{a}_l}{\widehat{\sigma}_l}\right),\\
		\text{где } \sum\limits_{j=1}^{k\cdot\widehat{k}}\widehat{p}_j=1, \, \widehat{p}_j\geqslant0, \, \widehat{a}_j\in\R, \, \widehat{\sigma}_j>0, \notag
		\end{gather}
		\begin{gather}
		\widehat{p}_{(r-1)\widetilde{k}+j}=p_r\widetilde{p}_j, \, 
		\widehat{a}_{(r-1)\widetilde{k}+j}=a_r+\widetilde{a}_j, \, 
		\widehat{\sigma}^2_{(r-1)\widetilde{k}+j}=\sigma^2_r+\widetilde{\sigma}^2_j, \label{PZ}\\ %\notag\\
		r=\overline{1,k}, \, j=\overline{1,\widetilde{k}}, \notag 
		\end{gather}
		и $\Phi(x)$ -- функция распределения стандартного нормального закона. В данном случае, все величины в выражении~\eqref{XDistr}, включая и число компонент $k$, считаем неизвестными, а в выражении~\eqref{YDistr} -- предварительно оцененными, например, с помощью какой-либо модификации EM-алгоритма.
		
		Тогда, как показано в статье~\cite{gorshenin2019adaptive}, оценки параметров неизвестного распределения с.в. $X$~\eqref{XDistr} по оценкам~\eqref{PZ} с.в. $Z$~\eqref{ZDistr} определяются следующими соотношениями:
		\begin{gather}
		a_r=\widetilde{k}^{-1}\sum\limits_{j=1}^{\widetilde{k}}\left(\widehat{a}_{(r-1)\widetilde{k}+j}-\widetilde{a}_j\right), \label{aX}\\
		\sigma^2_r=\widetilde{k}^{-1}\sum\limits_{j=1}^{\widetilde{k}}\left(\widehat{\sigma}^2_{(r-1)\widetilde{k}+j}-\widetilde{\sigma}^2_j\right), \label{sX}\\
		p_r=\widetilde{k}^{-1}\sum\limits_{j=1}^{\widetilde{k}}\widehat{p}_{(r-1)\widetilde{k}+j}\cdot \widetilde{p}_j^{-1}. \label{pX}
		\end{gather}
		
%		\subsection{Ход работы}
%			\todo{опущено для краткости}
%		\subsection{Результаты}
%			\todo{опущено для краткости}
		
\section{Выделение компонент связности}
    На втором этапе данный метод был применен к экспериментальным данным, описывающим поведение турбулентных потоков тепла в между океаном и атмосферой, а также решена подзадача о выделении компонент связности из полученных СРС-методом оценок.
       
    В статье \cite{gorshenin2020stat} подробно описано теоретическое обоснование применения данного метода для статистического оценивания коэффициентов стохастического дифференциального уравнения Ланжевена.
	\subsection{Постановка задачи}
		В физике стохастическим дифференциальным уравнением (СДУ) Ланжевена принято называть следующее соотношение:
		\begin{equation}
		\label{LangevinEq}
		dX(t)=a(t)dt+b(t)dW,
		\end{equation}
		определяющее случайный процесс $X(t)$, где $W(t)$ -- винеровский процесс, а а коэффициенты $a(t)$ и $b(t)$ -- случайны. СДУ вида~\eqref{LangevinEq} широко используются, например, в задаче ассимиляции данных при анализе разномасштабной изменчивости геофизических переменных~\cite{belyaev2018optimal}. В финансовой математике известны специальные версии уравнения~\eqref{LangevinEq}. В частности, весьма популярна модель геометрического броуновского движения вида
		\begin{equation}
		\label{BrownMotionEq}
		dX(t)=aX(t)dt+b X(t)dW,
		\end{equation}
		где $a\in\mathbb{R}$, $b>0$. Известно много обобщений модели~\eqref{BrownMotionEq} c конкретными видами зависимости $a$ и $b$ от $X(t)$ и других случайных процессов, например модели Леланда~\cite{leland1985option}, Барлса--Сонера~\cite{barles1998option}, Хестона \cite{heston1993closed}, Кокса--Ингерсолла--Росса~\cite{cox2005theory}, Халла--Уайта~\cite{hull1987pricing} и другие так называемые модели стохастической волатильности (см. также~\cite{derman1994riding, dupire1994pricing, shiryaev2004}).
		
		При отсутствии априорной информации о <<физической>> структуре процесса $X(t)$ для успешного изучения и прогнозирования его эволюции первостепенную важность приобретает задача  статистического оценивания функциональных коэффициентов $a(t)$ и $b(t)$. В силу их случайности данная задача допускает как минимум две принципиально разные формулировки. Во-первых, можно пытаться найти (случайные же) оценки самих функций $a(t)$ и $b(t)$, то есть найти их точечные аппроксимации, и, во-вторых, пытаться статистически оценить распределения случайных величин $a(t)$ и $b(t)$. Во втором случае, зная какие-либо свойства этих коэффициентов, например структуру их функциональной зависимости от исходного процесса $X(t)$ (как в упомянутых выше моделях Леланда, Барлса--Сонера, Хестона, Кокса--Ингерсолла--Росса или Беляева), можно найти оценки числовых параметров этих моделей.
		\\
		
		Рассматривается второй вариант постановки задачи.
		
%		\subsection{Ход работы}
%			\todo{опущено для краткости}
%		\subsection{Результаты}
%			\todo{опущено для краткости}
        
\section{Расширение пространства признаков}
    На последнем, третьем этапе работы освещается вопрос использования рассматриваемого метода для выделения дополнительной информации и расширения за счет этого пространства признаков нейронных сетей для улучшения качества получаемого прогноза временных рядов. 
    
    \subsection{Постановка задачи}
	    Пусть даны измерения некоей величины $X$, составляющие временной ряд: $X_1, ..., X_N$. Требуется предложить и реализовать метод расширения пространства признаков на основе изучаемого адаптивного метода. Целью является улучшение точности прогнозирования данного временного ряда с помощью нейронных сетей на среднюю длину окна (около 30 измерений).
	    
	    Рассматривается следующий набор архитектур нейронных сетей: LSTM-сети, Feedforward и Deep Feedforward сети, CNN-сети. Для каждой архитектуры планируется провести обучение для трех вариантов представления пространства признаков:
	    \begin{itemize}
	    	\item Необогащенное пространство признаков -- при получении предсказания на вход сети подается лишь окно с предшествующими искомому промежутку значениями величины $X$ (так называемые лаги временного ряда),
	    	\item Обогащенное моментами пространство признаков -- к входному вектору нейросети добавляются заранее посчитанные моменты выборки,
		    \item Обогащенное моментами и компонентами пространство признаков -- к входному вектору добавляются заранее посчитанные моменты выборки, а также параметры компонент, оцененные при помощи адаптивного метода и выделения связности, описанных в предыдущих разделах.
	    \end{itemize}
    
    \subsection{Ход работы}
	    Данный этап пока не завершен и находится в процессе доработки.
		 
        
\bibliographystyle{gost780u.bst} % Для соответствия требованиям об оформлении списка литературы
\raggedright
\bibliography{references}

% Раскомментируйте, если нужно приложение
% \appendix

% \cleardoublepage \phantomsection
% \section*{Приложение}
% \addcontentsline{toc}{section}{Приложение}

\end{document}
